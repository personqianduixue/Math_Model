\section{Assumptions}
\begin{itemize}
  \item[$\diamond$] \textbf{Assumption 1:} The wood decomposed by fungi is an ideal cylinder, and the fungi are distributed on the surface of the wood.
        \begin{itemize}
          \item[$\hookrightarrow$] \textbf{Justification:} Symmetrical the decomposition environment of the fungi as much as possible to facilitate the connection of this process with the Gauss theorem and Gauss surface in electromagnetic.
        \end{itemize}
  \item[$\diamond$] \textbf{Assumption 2:} The decomposition rate of the fungi is only related to its growth rate, tolerance to moisture and environment temperature.
        \begin{itemize}
          \item[$\hookrightarrow$] \textbf{Justification:} In the real natural environment, fungi are susceptible to natural disasters. We limit the determination of the factors affecting the decomposition rate of fungi in advance to facilitate subsequent analysis.
        \end{itemize}
  \item[$\diamond$] \textbf{Assumption 3:} After the part of Fungi Selection, we obtain five virtual fungal species, which are typical and representative.
        \begin{itemize}
          \item[$\hookrightarrow$] \textbf{Justification:} There are thousands of species of fungi, hence, it does not have much practical significance to study a specific species of fungi. The five fungi we set can be analyzed and studied more comprehensively.
        \end{itemize}
  \item[$\diamond$] \textbf{Assumption 4:} The Logistic model is also valid when studying microorganisms, such as fungi.
        \begin{itemize}
          \item[$\hookrightarrow$] \textbf{Justification:} Logistic model is generally used to model changes in common animals and plants populations. Since fungi are also biological, we will use analogy reasoning to apply it to the study of fungi.
        \end{itemize}
  \item[$\diamond$] \textbf{Assumption 5:} The study on the interactions of the two fungal species combination could show that of more fungal species combination.
        \begin{itemize}
          \item[$\hookrightarrow$] \textbf{Justification:} In a piece of natural land, although there are more than two types of fungi, the interaction of multiple fungi can be understood as the superimposed effect of multiple combinations.
        \end{itemize}
  \item[$\diamond$] \textbf{Assumption 6:} Fungi play a major role in the decomposition of organic matter in the certain carbon cycle.
        \begin{itemize}
          \item[$\hookrightarrow$] \textbf{Justification:} In biology, the decomposition process of organic matter is completed by microorganisms, including fungi, bacteria and so on. Since this article only studies fungi, this process is simplified.
        \end{itemize}
\end{itemize}