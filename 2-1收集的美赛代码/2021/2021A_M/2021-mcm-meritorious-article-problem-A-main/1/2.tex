\subsection{Restatement of Problem}
With basic knowledge of the earth's carbon circle and the vital agents in decomposing wood fibers called fungi, we analyze the relationship of two traits mentioned above and the rate of decomposition. Our team is assigned to solve the problems below.
\begin{itemize}
  \item[$\circledcirc$] \textbf{Problem 1:} Build the \textbf{decomposition model} of ground litter and woody fibers by calculating the \textbf{fungal activity factor}. Considering the interactions of fungi, fungal growth rate and moisture tolerance, the decomposition model would be optimized.
  \item[$\circledcirc$] \textbf{Problem 2:} Describe the \textbf{interactions} between different species of fungi, and characterize the \textbf{trends} of short- and long-term interactions.
  \item[$\circledcirc$] \textbf{Problem 3:} Analyze different \textbf{influences} on each isolate and fungal combinations \textbf{caused by different environments}, including arid, semi-arid, temperate, arboreal, and tropical rain forests.
  \item[$\circledcirc$] \textbf{Problem 4:} Assess the impact of the fungal communities' diversity on the earth' carbon circle efficiency, and explore the importance and role of fungal communities' \textbf{biodiversity} in the presence of different degrees of variability in the local environment.
\end{itemize}